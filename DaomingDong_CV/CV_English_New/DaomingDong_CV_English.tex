%% start of file `template.tex'.
%% Copyright 2006-2013 Xavier Danaux (xdanaux@gmail.com).
%
% This work may be distributed and/or modified under the
% conditions of the LaTeX Project Public License version 1.3c,
% available at http://www.latex-project.org/lppl/.


\documentclass[11pt,a4paper,roman]{moderncv}        % possible options include font size ('10pt', '11pt' and '12pt'), paper size ('a4paper', 'letterpaper', 'a5paper', 'legalpaper', 'executivepaper' and 'landscape') and font family ('sans' and 'roman')
% modern themes
\usepackage{mhchem}
\moderncvstyle{banking}                            % style options are 'casual' (default), 'classic', 'oldstyle' and 'banking'
\moderncvcolor{blue}                                % color options 'blue' (default), 'orange', 'green', 'red', 'purple', 'grey' and 'black'
%\renewcommand{\familydefault}{\sfdefault}         % to set the default font; use '\sfdefault' for the default sans serif font, '\rmdefault' for the default roman one, or any tex font name
%\nopagenumbers{}                                  % uncomment to suppress automatic page numbering for CVs longer than one page

% character encoding
\usepackage[utf8]{inputenc}                       % if you are not using xelatex ou lualatex, replace by the encoding you are using
%\usepackage{CJKutf8}                              % if you need to use CJK to typeset your resume in Chinese, Japanese or Korean

% adjust the page margins
\usepackage[scale=0.865]{geometry}
%\setlength{\hintscolumnwidth}{3cm}                % if you want to change the width of the column with the dates
%\setlength{\makecvtitlenamewidth}{10cm}           % for the 'classic' style, if you want to force the width allocated to your name and avoid line breaks. be careful though, the length is normally calculated to avoid any overlap with your personal info; use this at your own typographical risks...

\usepackage{import}

% personal data
\name{Daoming}{Dong}
\title{Curriculum Vitae}                               % optional, remove / comment the line if not wanted
\address{Department of Engineering, University of Cambridge}{}{}% optional, remove / comment the line if not wanted; the "postcode city" and and "country" arguments can be omitted or provided empty
\phone[mobile]{+44 7526060569}                   % optional, remove / comment the line if not wanted
% \phone[fixed]{01234 123456}                    % optional, remove / comment the line if not wanted
%\phone[fax]{+3~(456)~789~012}                      % optional, remove / comment the line if not wanted
\email{dd511@cam.ac.uk}                               % optional, remove / comment the line if not wanted
% \social[linkedin][www.linkedin.com/pub/stefano-bragaglia/51/22/b43/]{stefano-bragaglia} %Commented out as used above instead                         % optional, remove / comment the line if not wanted
\social[linkedin][https://www.linkedin.com/in/daoming-dong-684948b7/]{Daoming Dong}
\social[skype][live:dongdaoming@hotmail.com]{dongdaoming}                             % optional, remove / comment the line if not wanted
\social[gitlab][gitlab.com/DDMichael]{DDMichael}
\social[github][github.com/DDMichael]{DDMichael}

% \homepage{www.myname.webs.com}                         % optional, remove / comment the line if not wanted
%\extrainfo{additional information}                 % optional, remove / comment the line if not wanted
%\photo[64pt][0.4pt]{picture}                       % optional, remove / comment the line if not wanted; '64pt' is the height the picture must be resized to, 0.4pt is the thickness of the frame around it (put it to 0pt for no frame) and 'picture' is the name of the picture file
%\quote{Some quote}                                 % optional, remove / comment the line if not wanted

% to show numerical labels in the bibliography (default is to show no labels); only useful if you make citations in your resume
%\makeatletter
%\renewcommand*{\bibliographyitemlabel}{\@biblabel{\arabic{enumiv}}}
%\makeatother
%\renewcommand*{\bibliographyitemlabel}{[\arabic{enumiv}]}% CONSIDER REPLACING THE ABOVE BY THIS

% bibliography with mutiple entries
%\usepackage{multibib}
%\newcites{book,misc}{{Books},{Others}}
%----------------------------------------------------------------------------------
%            content
%----------------------------------------------------------------------------------
\begin{document}
%\begin{CJK*}{UTF8}{gbsn}                          % to typeset your resume in Chinese using CJK
%-----       resume       ---------------------------------------------------------
\makecvtitle
\vspace*{-1.5cm}

% \vspace*{0.4cm}
\section{Education}

\begin{itemize}
	\item{\cventry{2018--present}{PhD in Engineering }{University of Cambridge}{Cambridge, UK}{}{}}
	\item{\cventry{2016--2017}{MSc Advanced Materials Science and Engineering }{Imperial College London}{London, UK}{First class (75)}{}}
	\item{\cventry{2014--2016}{BEng (Hons) Electronics}{University of Liverpool}{Liverpool, UK}{First class with honours (75)}{}}  % arguments 3 to 6 can be left empty
	\item{\cventry{2012--2014}{BEng (Hons) Electronics Science and Engineering}{Xi'an Jiaotong Liverpool University}{Suzhou, China}{Top 1 (73) on progression to UoL}{}}
\end{itemize}

\section{Work Experience}
% \vspace{5pt}
\cventry{05/2018--05/2019}{VividQ Ltd.}{Research Consultant}{Cambridge, UK}{}{
\begin{itemize}
	\item Hardware and firmware design. Paid part time.
\end{itemize}}

% \vspace{5pt}
\cventry{06/2014--08/2014}{Department of Electrical Engineering, Xi'an Jiaotong University}{Research Assistant}{Suzhou, China}{}{
\begin{itemize}
	\item Supervisor: Dr. Derek Gray
	\item Power electronics circuit design and simulation via NI Multisim. Paid full time.
\end{itemize}}

\section{Project Portfolio}
% \vspace{5pt}
\cventry{01/2018--Present}{PhD Project}{Hardware implementations of 3D computer generated holography}{University of Cambridge}{}{
\begin{itemize}
	\item Supervisor: Prof. Timothy D. Wilkinson
	\item \textbf{Focus}: Investigate and implement the method to accelerate the CGH generation process using configurable heterogeneous hardwares including FPGA-SOC or FPGA-GPU system.
	\item PCB design, FPGA design, Matlab simulation and optical system set up.
% \item The purpose of this research is to investigate the capacitance-temperature behaviour of BCZT thin film. This material has the potential to be utilized in the next generation thermoelectric converter, and the core principle is the change in capacitance due to ambient temperature.
% \item \textbf{Conclusion:} For 400 nm BCZT film on MgO, a clear phase transition observed, and the Curie temperature is around $90^\circ C$. For 100 nm BCZT film on MgO, the strain shift the Curie temperature to a higher range. 
\end{itemize}}

\cventry{08/2018--Present}{PhD Side Project}{Awesome Board}{University of Cambridge}{}{
\begin{itemize}
	\item Supervisor: Prof. Timothy D. Wilkinson
	\item \textbf{Focus}: Develop a customized driver board for interfacing a high speed ferroelectric spatial light modulator.
	\item The board uses a low cost Lattice FPGA to communicate and transfer data between the PC and the SLM, it also features the USB3.0 and USB2.0 connectivity
	\item This mini-project was granted with two awards, the CAPE Acorn fund and the biomakers award.
	\item PCB design, FPGA design and system integration.
% \item The purpose of this research is to investigate the capacitance-temperature behaviour of BCZT thin film. This material has the potential to be utilized in the next generation thermoelectric converter, and the core principle is the change in capacitance due to ambient temperature.
% \item \textbf{Conclusion:} For 400 nm BCZT film on MgO, a clear phase transition observed, and the Curie temperature is around $90^\circ C$. For 100 nm BCZT film on MgO, the strain shift the Curie temperature to a higher range. 
\end{itemize}}

% \vspace{5pt}
\cventry{12/2016--09/2017}{MSc Project}{Investigate the C-T relationship of thin film BCZT material}{Imperial College London}{}{
\begin{itemize}
	\item Supervisor: Dr. Peter K. Petrov
	\item \textbf{Focus}: dielectric thin film device fabrication and characterization
	\item Full clean room fabrication experience including sample preparation, spin coating, photolithography, pulse laser deposition (PLD), evaporation and reactive ion etching.
	\item Thin film devices characterization: surface analysis with Dektak profilometer, scanning electron microscopy (SEM), atomic force microscopy (AFM), x-ray diffraction (XRD) and probe station with semiconductor analyzer; electrical property investigation by the use of probe station with semiconductor analyzer.
% \item The purpose of this research is to investigate the capacitance-temperature behaviour of BCZT thin film. This material has the potential to be utilized in the next generation thermoelectric converter, and the core principle is the change in capacitance due to ambient temperature.
% \item \textbf{Conclusion:} For 400 nm BCZT film on MgO, a clear phase transition observed, and the Curie temperature is around $90^\circ C$. For 100 nm BCZT film on MgO, the strain shift the Curie temperature to a higher range. 
\end{itemize}}

% \vspace{5pt}
\cventry{09/2015--06/2016}{BEng Project}{Transparent electronics - thin film transistors}{University of Liverpool}{}{
\begin{itemize}
	\item Supervisor: Prof. Steve Hall
	\item \textbf{Focus}: Investigate the current transport of novel oxide semiconductor thin film transistor for transparent thin film electronics.
	\item Clean room fabrication and measurement experience, MatLab modeling.
\end{itemize}}
% \vspace{0.3cm}
% % \cventry{10/2015--11/2015}{Design Project}{Design a UART between PC and FPGA via RS232 port}{University of Liverpool}{}{
% % \begin{itemize}
% % \item Conduct literature review on the Universal Asynchronous Receiver/Transmitter protocol;
% % \item Draw the Block diagram for transmitter and receiver and also the ASM charts for every individual module (controller, baud counter, bit counter and shift register);
% % \item Code the design by Verilog with the aid of the ASM charts, then simulate the design using ModelSim simulator;
% % \item Download the design into FPGA board to debug and test by connecting the board with PC and using the RS232.
% % \item \textbf{Conclusion:} The FPGA and PC can achieve bidirectional 7-bit ascii data transmission.
% % \end{itemize}
% % }
% \vspace{0.3cm}
% \cleardoublepage
% \cventry{12/2016--09/2017}{MSc Project}{Investigate the C-T relationship of thin film BCZT material}{Imperial College London}{}{
% \begin{itemize}
% \item Conduct literature review on various thermoelectric devices;
% \item Use the PLD system to deposit the BCZT with desired stoichiometry on the MgO substrate, then deposit metal and do photolithography and etching for the sample;
% \item Use the LCR meter or semiconductor analyser to measure the capacitance temperature relationship of the patterned thin film;
% \item \textbf{Conclusion:} Still under investigation
% \end{itemize}
\section{Additional Skills and Achievements}
\subsection{Subject Related}
\begin{itemize}
	\item \textbf{Scientific computing and modeling:} Proficient in Matlab. Know well in Python with data analysis packages.
	\item \textbf{Programming language:} Medium in C/C++. Know well in Python. Know well in CUDA for parallel computing.
	\item \textbf{Hardware description language:} Proficient in Verilog. Know well in SystemVerilog and VHDL. Experience in coding communication protocols (UART and SPI) and arithmetics unit (2D fast Fourier Transform).
	\item \textbf{Field programmable gate array design:} Proficient in Intel Quartus Prime design suite and Lattice iCEcube2 design suite. Know well in Xilinx Vivado and ISE design suite.
	\item \textbf{Printed circuit board design:} Proficient in Altium designer. Know well Eagle. Experience in design high speed PCB with differential signaling and FPGA.
	\item \textbf{Holographic projection system set up:} Experience in setting up a holographic projection system with Throlab components.
	\item \textbf{Instruction set architecture:} Basic in ARM 7.
	\item \textbf{Operating systems:} Proficient in MacOS and Linux (Ubuntu, CentOS, etc.).
\end{itemize}
\subsection{IT Skills}
\begin{itemize}
	\item \textbf{Web development:} Know well in HTML, CSS, Javascript and ruby, basic in ruby on rails framework and MongoDB database.
	\item \textbf{Adobe Family:} Proficient in Lightroom and Photoshop. Know well in Illustrator and After Effect.
	\item \textbf{Photography:} Proficient in portrait and landscape photography and post-editing.
	\item \textbf{Others:} *nix command line, Git, \LaTeX.
\end{itemize}
% \subsection{Languages}
% \begin{itemize}
% \item \textbf{Chinese:} Native
% \item \textbf{Cantonese:} Conversational
% \item \textbf{English:} Fluent
% \end{itemize}
\subsection{Achievements}
\begin{itemize}
\item{\cventry{May, 2019}{University of Cambridge}{Biomaker award}{Cambridge, UK}{EPSRC}{}}

\item{\cventry{April, 2019}{University of Cambridge}{CAPE Acorn award}{Cambridge, UK}{Department of Engineering}{}}

% \item{\cventry{October, 2017}{Imperial College London}{MSc with distinctions}{London, UK}{}{}}

% \item{\cventry{July, 2016}{University of Liverpool}{BEng (Hons) First class with honours}{Liverpool, UK}{}{}}

% \item{\cventry{July, 2016}{Xi'an Jiaotong Liverpool University}{BEng (Hons) First class with honours}{Suzhou, China}{}{}}
\item{\cventry{September, 2019}{Microsoft on Edx (DEV210.3x)}{Advanced C++}{Online}{}{}}
\item{\cventry{September, 2019}{University of Illinois at Urbana-Champaign on Coursera}{Object-oriented Data Structures in C++ }{Online}{}{}}
\item{\cventry{August, 2016}{John Hopkins University on Coursera}{Rails with Active Record and Action Pack}{Online}{}{}}
\item{\cventry{August, 2016}{John Hopkins University on Coursera}{HTML, CSS, and Javascript for Web Developers}{Online}{}{}}
\item{\cventry{July, 2016}{John Hopkins University on Coursera}{Ruby on Rails: An Introduction}{Online}{}{}}

\item{\cventry{June, 2014}{University of Liverpool}{50\% reduction in tuition fees of University of Liverpool (top 5\%)}{Liverpool, UK}{}{}}
\item{\cventry{July 31$^\text{st}$, 2013}{Mount Kilimanjaro National Park}{Certificate of successful summit bid of Mt.Kilimanjaro in Africa (5895m)}{Arusha, Tanzania}{}{}}
\item{\cventry{June -- August, 2013}{University of Dar es Salaam}{AIESEC volunteer at Library Project}{Dar es Salaam, Tanzania}{}{}}
\item{\cventry{January -- Febuary, 2013}{Delhi IIT}{AIESEC volunteer at at Project Umeed at AIESEC Delhi IIT}{Delhi, India}{}{}}
\end{itemize}

\section{Publication Lists}
[1] \textsc{Fixed-Point Accuracy analysis of 2D FFT for the creation of computer generated hologram}\\
\textbf{D. Dong}, Y. Wang, P. Christopher, A. Kadis and T. Wilkinson. 2019 IEEE Global Conference on Signal and Information Processing.

[2] \textsc{Computer Hologram Generation With One-Step Phase-Retrieval Using a Digital Signal Processor}\\
Y. Wang, \textbf{D. Dong}, P. Christopher, A. Kadis and T. Wilkinson. 2019 IEEE Global Conference on Signal and Information Processing.

[3] \textsc{Improving Holographic Search Algorithms using Sorted Pixel Selection}\\
P. Christopher, J. Lake, \textbf{D. Dong}, H. Joyce and T. Wilkinson.  J. Opt. Soc. Am. A 36, 1456-1462 (2019)

[4] \textsc{Hardware Implementations On Computer Generated Holography: A Review}\\
Y. Wang, \textbf{D. Dong}, P. Christopher, A. Kadis, R. Mouthaan, F. Yang and T. Wilkinson. \textit{In Submission,} 2019.

[5] \textsc{Lookup tables for phase randomisation in hardware generated holograms}\\
P. Christopher, Y. Wang, \textbf{D. Dong}, R. Mouthaan, A. Kadis and T. Wilkinson. \textit{In submission,} 2019.

\end{document}


%% end of file `template.tex'.
